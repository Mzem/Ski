\documentclass[a4]{article}
\usepackage[utf8]{inputenc}
\usepackage[french]{babel}
\usepackage{listings}
\usepackage{color}

\definecolor{mygreen}{rgb}{0,0.6,0}
\definecolor{mygray}{rgb}{0.5,0.5,0.5}
\definecolor{mymauve}{rgb}{0.58,0,0.82}

\lstset{
  backgroundcolor=\color{white},   % choose the background color; you must add \usepackage{color} or \usepackage{xcolor}
  basicstyle=\footnotesize,        % the size of the fonts that are used for the code
  breakatwhitespace=false,         % sets if automatic breaks should only happen at whitespace
  breaklines=true,                 % sets automatic line breaking
  captionpos=b,                    % sets the caption-position to bottom
  commentstyle=\color{mygreen},    % comment style
  deletekeywords={...},            % if you want to delete keywords from the given language
  escapeinside={\%*}{*)},          % if you want to add LaTeX within your code
  extendedchars=true,              % lets you use non-ASCII characters; for 8-bits encodings only, does not work with UTF-8
  frame=L,	                       % adds a frame around the code
  keepspaces=true,                 % keeps spaces in text, useful for keeping indentation of code (possibly needs columns=flexible)
  keywordstyle=\color{blue},       % keyword style
  language=C,                 	   % the language of the code
  otherkeywords={*,...},           % if you want to add more keywords to the set
  numbers=none,                    % where to put the line-numbers; possible values are (none, left, right)
  numbersep=5pt,                   % how far the line-numbers are from the code
  numberstyle=\tiny\color{mygray}, % the style that is used for the line-numbers
  rulecolor=\color{black},         % if not set, the frame-color may be changed on line-breaks within not-black text (e.g. comments (green here))
  showspaces=false,                % show spaces everywhere adding particular underscores; it overrides 'showstringspaces'
  showstringspaces=false,          % underline spaces within strings only
  showtabs=false,                  % show tabs within strings adding particular underscores
  stepnumber=2,                    % the step between two line-numbers. If it's 1, each line will be numbered
  stringstyle=\color{mymauve},     % string literal style
  tabsize=2,	                   % sets default tabsize to 2 spaces
  title=\lstname                   % show the filename of files included with \lstinputlisting; also try caption= instead of title
}
%gestion des caractères latins
\lstset{literate=
  {á}{{\'a}}1 {é}{{\'e}}1 {í}{{\'i}}1 {ó}{{\'o}}1 {ú}{{\'u}}1
  {Á}{{\'A}}1 {É}{{\'E}}1 {Í}{{\'I}}1 {Ó}{{\'O}}1 {Ú}{{\'U}}1
  {à}{{\`a}}1 {è}{{\`e}}1 {ì}{{\`i}}1 {ò}{{\`o}}1 {ù}{{\`u}}1
  {À}{{\`A}}1 {È}{{\'E}}1 {Ì}{{\`I}}1 {Ò}{{\`O}}1 {Ù}{{\`U}}1
  {ä}{{\"a}}1 {ë}{{\"e}}1 {ï}{{\"i}}1 {ö}{{\"o}}1 {ü}{{\"u}}1
  {Ä}{{\"A}}1 {Ë}{{\"E}}1 {Ï}{{\"I}}1 {Ö}{{\"O}}1 {Ü}{{\"U}}1
  {â}{{\^a}}1 {ê}{{\^e}}1 {î}{{\^i}}1 {ô}{{\^o}}1 {û}{{\^u}}1
  {Â}{{\^A}}1 {Ê}{{\^E}}1 {Î}{{\^I}}1 {Ô}{{\^O}}1 {Û}{{\^U}}1
  {œ}{{\oe}}1 {Œ}{{\OE}}1 {æ}{{\ae}}1 {Æ}{{\AE}}1 {ß}{{\ss}}1
  {ű}{{\H{u}}}1 {Ű}{{\H{U}}}1 {ő}{{\H{o}}}1 {Ő}{{\H{O}}}1
  {ç}{{\c c}}1 {Ç}{{\c C}}1 {ø}{{\o}}1 {å}{{\r a}}1 {Å}{{\r A}}1
  {€}{{\EUR}}1 {£}{{\pounds}}1
}
%definition d'un syle pour les documents texte
\lstdefinestyle{txt}{
	frame=none,
	numbers=none,
	stringstyle=\color{black},
}

\author{Younes Ben Yamna - Malek Zemni}
\title{Rapport - Projet théorie des graphes}
\date{\today}

\begin{document}
\maketitle

	\section{Introduction}
			L'idée principale de ce projet est de trouver le plus court chemin entre deux sommets d'un graphe orienté.\\
			Le graphe qui nous a été fourni - la carte de l'Alpe d'Huez - correspond à un graphe orienté muni d'une fonction de pondération positive, 
			c'est à dire que tous les arcs ont un poids positif. Ce graphe n'est pas un graphe simple à première vue (existence de plusieurs arcs 
			entre les même sommets), mais on pourra voir dans les sections suivantes qu'on pourra le considérer ainsi.
			
	\section{Graphe}
			La première étape de ce projet était d'analyser la carte de la station de ski fournie et d'en tirer le graphe correspondant. 
			On a donc considéré les pistes de ski et les remontées mécaniques comme des arcs, et les points d'intersection, de déprat et 
			d'arrivée de ces pistes comme des sommets.
			
		\subsection{Sommets du graphe}
			\paragraph{Considération des sommets :\\}
			On considère comme sommets du graphe, les points de déprat et d'arrivée des pistes et les points où se croisent plusieurs de ces pistes. 
			La carte fournie a été simplifée puisque les sommets correspondant à des points où se croisent plusieurs pistes sont en fait des zones 
			(et non des points) où se croisent ces pistes.\\
			Les noms des sommets sont quant à eux choisis par rapport au nom de la zone où se trouve le sommet. 
			Si la zone ne porte pas de nom, le nom du sommet sera choisi par rapport au nom de la piste ou de la remontée la plus proche.
			\paragraph{Indices des sommets :\\}
			Chaque sommet a été repéré par un indice allant de 0 à 44 : \textbf{on a donc 45 sommets au total}.
			Voici la liste complète des sommets, ainsi que leurs indices respectifs :
			\lstinputlisting[firstline=4,style=txt,title=sommets.txt]{../data/sommets.txt}
			
		\subsection{Arcs du graphe}
			\paragraph{Considération des arcs :\\}
			Les arcs du graphe correspondent bien évidemment aux pistes de ski. On a alors deux types d'acrs : des \textit{descentes} et des
			\textit{remontées}.\\
			Un simplification du graphe a été effectuée : s'il existe plusieurs arcs de meme extrémité de départ et d'arrivée, 
			on ne retient que l'arc de poids le plus faible. La raison est que pour calculer un plus court chemin, l'arc de poids faible est 
			le seul qui va etre pris en compte, peu importe le nombre des autres arcs liant ces mêmes sommets.
			\paragraph{Couleur et poids des arcs :\\}
			Les différents types d'arcs ont chacun leur propre couleur qui a été repérée par un indice. Cette coloration influe directement dans 
			le calcul du poids de l'arc pour le \textbf{cas d'un skieur débutant} :
			\begin{itemize}
			\item{0 - vert : descente de poids normal}
			\item{1 - bleu : descente dont le poids est multiplié par 2}
			\item{2 - rouge : descente dont le poids est multiplié par 3}
			\item{3 - noir : descente dont le poids est multiplié par 4}
			\item{4 - noir : remontée dont le poids dépend de son type (télésiège, téléski, funitel, etc.)}
			\end{itemize}
			Voici la fonction qui permet la conversion du temps de parcours de l'arc et de sa couleur en poids (l'entier experience vaut 0 pour un
			expert et 1 pour un débutant) :
			\lstinputlisting[firstline=8,lastline=41,title=graphe.c]{../src/graphe.c}
			
			\paragraph{Indices des arcs :\\}
			Chaque arc dans le graphe est repéré par un indice.\\
			Les descentes sont repérées par des indices allant de 0 à 56 : on a donc 57 descentes.\\
			Les remontées sont repérées par des indices allant de 0 à 42 : on a donc 43 remontées.\\
			Remarque : on ajoute 100 à chaque indice de remontée pour les différencier des descentes.\\
			On a alors 57 descentes et 43 remontées, ce qui fait \textbf{100 arcs au total}.\\
			Voici la liste des arcs, ainsi que leurs indices respectifs :
			\lstinputlisting[firstline=12,style=txt,,title=arcs.txt]{../data/arcs.txt}
		
		\subsection{Fichier du graphe}
			Un fois les sommets et les arcs considérés, on a pu constituer notre graphe et le rentrer dans un fichier texte en respectant ce format :
			\begin{itemize}
			\item{le 1er entier est l'indice du sommet de depart}
			\item{le 2ème entier est l'indice du sommet d'arrivee}
			\item{le 3ème entier est l'indice de l'arc liant le sommet de départ au sommet d'arrivée}
			\item{le 4ème entier est le temps de parcours (mesuré directement sur la carte) de cet arc}
			\item{le 5ème entier est la couleur de cet arc}
			\end{itemize}
			Voici un petit extrait de ce fichier :
			\lstinputlisting[firstline=5,lastline=7,title=graphe.txt]{../data/graphe.txt}
			
	\section{Structure}
			La strucutre utilisée pour représenter le graphe est une matrice d'adjascence. Cette matrice est en fait un tableau à deux dimensions 
			dont la première dimension (les lignes) correpond à l'indice du sommet de départ, et la deuxième dimension (les colonnes) correspond 
			à l'indice du sommet d'arrivée.\\
			Les éléments de ce tableau ont un type personnalisé \textbf{Arc} qui stocke toutes les informations relatives à un arc du graphe :
			\lstinputlisting[firstline=14,lastline=22,title=definitions.h]{../src/definitions.h}
			
			Le graphe a été déclaré comme ceci (V étant le nombre de sommets) :
			\lstinputlisting[firstline=24,lastline=24,title=definitions.h]{../src/definitions.h}
			Remarque : ce tableau a été déclaré comme variable globale par soucis de clareté du code : on a préféré le garder ainsi 
			puisqu'à la fin, ce projet est un projet d'algorithmique et non de programmation.
			
			\paragraph{Lecture du fichier du graphe :\\}
			Une fois la structure du graphe bien définie, on a pu créer une fonction pour le remplissage de la variable G (le graphe) à partir du
			fichier texte :
			\lstinputlisting[firstline=357,title=graphe.c]{../src/graphe.c}
	
	\section{Recherche du plus court chemin}
			On va maintentant s'interésser à la recherche d'un plus court chemin entre un sommet de départ et un sommet d'arrivée.\\
			La fonction principale qui permet de réaliser cette manipulation est la fonction \textbf{dijkstra} qui est une application 
			directe de l'algorithme de meme nom.\\
			Cet algorithme a été préféré à d'autres pour sa facilité d'implémentation, d'autant plus que le graphe fourni en entrée remplit les 
			conditions (arcs de poids positifs).\\
			La fonction dijkstra va se charger de remplir un tableau de pères \textbf{pere[V]} préalablement initialisé à -1 : 
			\textit{pere[x] = y} veut dire que le sommet d'indice x a pour père le sommet d'indice y. Elle va donc calculer les plus courts chemins 
			à partir d'un sommet racine vers tous les sommets du graphe qui lui sont accessibles (construction d'une arboresence des plus courts 
			chemins).\\
			Voici la fonction commentée étape par étape :
			\lstinputlisting[firstline=23,title=dijkstra.c]{../src/dijkstra.c}
			
			\paragraph{Plus court chemin : \\}
			Comme on l'a dit précedemment, la fonction dijkstra va calculer tous les plus courts chemins à partir d'un sommet racine. Cependant, 
			notre objectif ici est de rechercher le plus court chemin entre un sommet de départ et un sommet d'arrivée précis.\\
			L'idée est d'exploiter le tableau pere[] en le parcourant à partir du sommet d'arrivée et en remontant dans la hiérarchie des pères 
			jusqu'à retrouver les sommet de départ. Cette séquence sera ensuite stockée dans une liste chainées (plus performante ici qu'un tableau
			car on ne connait pas la taille du chemin) pour etre finalement affichée.\\
			Voici la fonction qui s'occupe de cette manipulation (elle comporte quelques éléments de l'affichage graphique, n'y portez pas attention) 
			:
			\lstinputlisting[firstline=53,lastline=103,title=fonctions.c]{../src/fonctions.c}
	
			
	\section{Conclusion}
			En guise de conclusion, on peut dire que ce projet nous a permis d'avoir un très bon aprecu sur les champs d'application des algorithmes 
			liés à la théorie des graphes et nous a donné l'occasion de plonger au coeur de l'action, surtout pour un problème qui nous 
			est très familier aujourd'hui.\\
			
			Le code source de ce projet est "gratuitement" disponible sur github : \textbf{https://github.com/Mzem/Ski}
		

\end{document}


