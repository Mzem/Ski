\documentclass[a4]{article}
\usepackage{inputenc}
\usepackage[french]{babel}
\usepackage{listings}
\usepackage{color}


\definecolor{mygreen}{rgb}{0,0.6,0}
\definecolor{mygray}{rgb}{0.5,0.5,0.5}
\definecolor{mymauve}{rgb}{0.58,0,0.82}

\lstset{ %
  backgroundcolor=\color{white},   % choose the background color; you must add \usepackage{color} or \usepackage{xcolor}
  basicstyle=\footnotesize,        % the size of the fonts that are used for the code
  breakatwhitespace=false,         % sets if automatic breaks should only happen at whitespace
  breaklines=true,                 % sets automatic line breaking
  captionpos=b,                    % sets the caption-position to bottom
  commentstyle=\color{mygreen},    % comment style
  deletekeywords={...},            % if you want to delete keywords from the given language
  escapeinside={\%*}{*)},          % if you want to add LaTeX within your code
  extendedchars=true,              % lets you use non-ASCII characters; for 8-bits encodings only, does not work with UTF-8
  frame=single,	                   % adds a frame around the code
  keepspaces=true,                 % keeps spaces in text, useful for keeping indentation of code (possibly needs columns=flexible)
  keywordstyle=\color{blue},       % keyword style
  language=Octave,                 % the language of the code
  otherkeywords={*,...},           % if you want to add more keywords to the set
  numbers=left,                    % where to put the line-numbers; possible values are (none, left, right)
  numbersep=5pt,                   % how far the line-numbers are from the code
  numberstyle=\tiny\color{mygray}, % the style that is used for the line-numbers
  rulecolor=\color{black},         % if not set, the frame-color may be changed on line-breaks within not-black text (e.g. comments (green here))
  showspaces=false,                % show spaces everywhere adding particular underscores; it overrides 'showstringspaces'
  showstringspaces=false,          % underline spaces within strings only
  showtabs=false,                  % show tabs within strings adding particular underscores
  stepnumber=2,                    % the step between two line-numbers. If it's 1, each line will be numbered
  stringstyle=\color{mymauve},     % string literal style
  tabsize=2,	                   % sets default tabsize to 2 spaces
  title=\lstname                   % show the filename of files included with \lstinputlisting; also try caption= instead of title
}

\author{Younes Ben Yamna - Malek Zemni}
\title{Rapport - Projet théorie des graphes}
\date{\today}

\begin{document}
\lstset{language=C}
\maketitle

	\section{Introduction}
			ici on parle de l''idee generale et comment on va faire pour le projet
	\section{Structures}
		les structures utilsisee sont :
			la structure arc qui represente un arc (comme son nom l'indique)\\
			la structure parcour qui contiend le poid de chaque sommet (par rapport au sommet de depart)
			 son indice et une variable parcouru qui indique si il est parcouru ou non (-1 ou 0)\\
			la structure antecedant qui contiend l'indice du sommet et son pere (le sommet par lequel on passe pour arrivera notre sommet le plus rapidement possible)
		\lstinputlisting[firstline=7]{src/definitions.h}
	\section{Graphe}
		\subsection{Sommets du graphe}
			\paragraph{Considération des sommets\\}
			On considère comme sommets les points où se croisent plusieurs pistes. La carte fournie a été simplifée, c'est à dire qu'on
			considère en fait comme sommets les zones (et non les points) où se croisent plusieurs pistes.\\
			Les noms des sommets sont donc choisis par rapport au nom de la zone où se trouve le sommet. 
			Si la zone ne porte pas de nom, le nom du sommet sera choisi par rapport au nom de la piste ou de la remontée la plus proche.
			\paragraph{Indices des sommets\\}
			Chaque sommet est repéré par un indice allant de 0 à 44 : on a donc 45 sommets au total.\\
			Voici la liste complète des sommets, ainsi que leurs indices respectifs :
			\lstinputlisting[firstline=13]{src/Sommets.txt}
		\subsection{Arcs du graphe}
			\paragraph{Considération des arcs\\}
			voici tout les arcs qu'on a considerer dans notre travail\\
			NB: si deux arcs relient deux sommet dans le meme sens on garde celui avec le poid le plus petit 
			\lstinputlisting[firstline=13]{src/Arcs.txt}
		\subsection{Fichier du graphe}
			le premier entier est le nom du sommet de depart (enfin l'indice qui represente ce nom)\\
			le deuxieme entier est le nom du sommet d'arrivee(enfin l'indice qui represente ce nom)\\
			le troisieme entier est le nom de l'arc (enfin l'indice qui represente ce nom)\\
			le quatrieme entier est le poid de cet arc (sans considerer la couleur)\\
			le cinqieme entier est la couleur de cet arc\\
			\lstinputlisting[firstline=7,lastline=15]{src/graphe.txt}
		\subsection{lecture du graphe}
			la fonction utilisee est elle fait ...
			\lstinputlisting[firstline=419,lastline=446]{src/graphe.c}
	\section{les fonctions}
		\subsection{get experience}
			c'est la fontion qui
			\lstinputlisting[firstline=9,lastline=21]{src/fonctions.c}
		\subsection{calcul poid}
			c'est la fontion qui
			\lstinputlisting[firstline=23,lastline=55]{src/fonctions.c}
	\section{Algorithme de Dijkstra}
		\subsection{intro}
			la on parle du principe comment et pourquoi
		\subsection{initialisation des tableaux}
			\lstinputlisting[firstline=7,lastline=26]{src/dijkstra.c}
		\subsection{recherche des peres et des fils}
			\lstinputlisting[firstline=28,lastline=55]{src/dijkstra.c}
		\subsection{l'algo de dijkstra}
			\lstinputlisting[firstline=57]{src/dijkstra.c}
	\section{Conclusion et difficultées}
		

\end{document}


