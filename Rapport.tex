\documentclass[a4]{article}
\usepackage[utf8]{inputenc}
\usepackage[french]{babel}

\author{Younes Ben Yamna - Malek Zemni}
\title{Rapport - Projet théorie des graphes}
\date{\today}

\begin{document}
\maketitle

	\section{Introduction}
	\section{Graphe}
		\subsection{Sommets du graphe}
			\paragraph{Considération des sommets\\}
			On considère comme sommets les points où se croisent plusieurs pistes. La carte fournie a été simplifée, c'est à dire qu'on
			considère en fait comme sommets les zones (et non les points) où se croisent plusieurs pistes.\\
			Les noms des sommets sont donc choisis par rapport au nom de la zone où se trouve le sommet. 
			Si la zone ne porte pas de nom, le nom du sommet sera choisi par rapport au nom de la piste ou de la remontée la plus proche.
			\paragraph{Indices des sommets\\}
			Chaque sommet est repéré par un indice allant de 0 à 44 : on a donc 45 sommets au total.\\
			Voici la liste complète des sommets, ainsi que leurs indices respectifs :
			\begin{itemize}
			\item 0 PIC BLANC
			\item 1 GROTTE DE GLACE
			\item 2 SOMMET 3060
			\item 3 SARENNE BASSE
			\item 4 CLOCHER DE MACLE
			\item 5 LAC BLANC
			\item 6 LIEVRE BLANC
			\item 7 PLAT DES MARMOTTES
			\item 8 MINE DE L'HERPIE
			\item 9 SOMMET 2100
			\item 10 SOMMET DES VACHETTES
			\item 11 SIGNAL DE L'HOMME
			\item 12 L'ALPETTE
			\item 13 COL DU COUARD
			\item 14 CASCADE
			\item 15 CLOS GIRAUD
			\item 16 MONFRAIS
			\item 17 SIGNAL
			\item 18 RIFNEL EXPRESS
			\item 19 CHALVET
			\item 20 AURIS EXPRESS
			\item 21 FONTFROIDE
			\item 22 LOUVETS
			\item 23 POUTRAN
			\item 24 CHAMP CLOTURE
			\item 25 STADE
			\item 26 SCHUSS
			\item 27 ALPE D'HUEZ
			\item 28 GRANDE SURE
			\item 29 ECLOSE
			\item 30 SURES
			\item 31 COL
			\item 32 AURIS EN OISANS
			\item 33 LA VILLETTE
			\item 34 VAUJANY
			\item 35 L'EVERSIN D'OZ
			\item 36 OZ EN OISANS
			\item 37 PETIT PRINCE
			\item 38 VILLAGE
			\item 39 MARONNE
			\item 40 VILLARD RECULAS
			\item 41 HUEZ
			\item 42 DOME DES PETITES ROUSSES
			\item 43 ALPAURIS
			\item 44 L'ALPETTE
			\end{itemize}
		\subsection{Arcs du graphe}
			\paragraph{Considération des arcs\\}
		\subsection{Fichier du graphe}
	\section{Structure}
	\section{Algorithme}
		\subsection{Lecture du graphe}
		\subsection{Affichage du graphe}
	\section{Conclusion}

\end{document}


